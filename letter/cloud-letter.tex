\documentclass[a4paper,11pt]{letter}
\usepackage[T1]{fontenc}
\usepackage[utf8]{inputenc}
\usepackage{lmodern}
\usepackage[brazilian]{babel}
\usepackage{hyperref}

\signature{ Prof. Dr. Fabio Kon }
\address{ Departamento de Ciência da Computação \\ Rua do Matão, 1010 \\ 05508-090 São Paulo  - SP \\ Brasil}

\begin{document}

\begin{letter}{ Diretor de informática \\ Vice-Reitoria Executiva de Administração (VREA) da Universidade }

\opening{ Caro diretor, }

Tomei conhecimento recentemente de que a nuvem da universidade está operacional e já com alguns usuários obtendo resultados positivos nesta. Assim, gostaria de indagá-lo quanto a possibilidade de aproveitar sua estrutura em prol de um projeto acadêmico que vem sendo desenvolvido sob minha orientação por alunos e ex-alunos do Departamento de Ciência da Computação do Instituto de Matemática e Estatística.

Sobre o projeto em questão, denominado Mezuro, que envolve desde alunos de doutorado até alunos de inicição científica. Com apoio de NAPSOL e CNPQ, o projeto está em vias de ser liberado para uso público assim que solucionarmos questões técnicas como a infraestrutura.

Ele pode ser encontrado atualmente no endereço (mas ainda não o divulgamos pois o atual servidor não tem capacidade de atender à demanda) \url{mezuro.org} sob a licensa AGPL. Lá é possível experimentar suas funcionalidades que tem como objetivo difundir métricas de código-fonte e analisar seus padrões de uso.

Assim, caso seja possível, os recursos que esperamos poder utilizar são por volta de uma máquina com alto poder de processamento para realizar o cálculo de métricas e outras duas de poder médio, uma para servir o sítio e outra como servindo de banco de dados. Da mesma forma, gostaríamos de poder associar um endereço IP público apenas ao servidor do sítio.

\closing{Atenciosamente,}

\end{letter}

\end{document}
