\documentclass[12pt]{article}

\usepackage{url}
\usepackage[brazil]{babel}
\usepackage[utf8]{inputenc}
\usepackage{float}
\usepackage{setspace}
\usepackage{tabu}

\begin{document}
  \title{Plano de Trabalho - Projeto Mezuro\\
         Eduardo Silva Araújo\\
         NUSP 8867461}

  \maketitle

  \section{Justificativa}
  Tendo em vista que o aluno Pedro Alves de Medeiros Scocco (NUSP 7558183) não tem a disponibilidade de tempo para conciliar o projeto com suas obrigações de disciplinas e trabalho de conclusão de curso na sua graduação e, portanto, não renovará seu vinculo com fim em 26 de Setembro de 2015.

  Para manter a trajetória descrita no documento \textit{Projeto Mezuro - Justificativa para Bolsistas} é importante manter o mesmo fluxo de trabalho com dois estagiários dedicados. Assim o aluno Eduardo, objetivo deste plano de trabalho, será um substituto à altura do aluno Pedro que o projeto perderá.

    \subsection{Qualificações}
    Já possui experiência no projeto Mezuro como contribuidor independente desde Julho de 2015, sendo assim um candidato muito bom para contribuir ao projeto uma vez que não terá que passar pela curva de aprendizado natural.

  \section{Atividades}
    Vale ressaltar que o projeto adota a metodologias ágeis como programação extrema. Portanto, para estas atividades o aluno terá mais uma pessoa junto. Além da supervisão de alunos mais experientes.

    \subsection{Atividades cotidianas}
      Atividades que serão exercidas todos os dias pelo aluno concorrentemente com o desenvolvimento de funcionalidades.

      \begin{itemize}
        \item Dar manutenção aos serviços do \url{mezuro.org}, resolvendo falhas que apareçam;
        \item Revisar os códigos dos colaboradores externos e internos do projeto.
      \end{itemize}

    \subsection{Novas Funcionalidades Previstas}\label{subsec:func-prev}
      O objetivo principal das funcionalidades pretendidas a serem desenvolvidas pelo aluno em conjunto com o restante do time de desenvolvimento tem o foco principal em melhorar a interação dos usuários finais com o sistema.

      \begin{enumerate}
        \item Criação de um subtipo de configuração para métricas de \textit{hotspot};
        \item Reestruturação da conexão de novas ferramentas coletoras de métricas para melhorar a manutenabilidade e encapsulamento;
        \item Incorporar a ferramenta \textit{CodeClimate OpenSource} para coleta de métricas para diversas linguagens inclusive PHP.
      \end{enumerate}

      \subsubsection{Cronograma}
        \begin{table}[H]
          \begin{tabu}{| c | c | c | c | c | c | c | c |}
            \hline
            \textbf{Mês} & Outubro & Novembro & Dezembro & Janeiro & Fevereiro & Março & Abril\\ \hline
            \textbf{Atividades} & 1 & 1 e 2 & 2 & 2 & 2 & 2 e 3 & 3\\ \hline
          \end{tabu}
          \caption{Cronograma das funcionalidades previstas numeradas de acordo com a lista em \ref{subsec:func-prev}}
        \end{table}

        Vale ressaltar que o estágio é de \textbf{6 meses} e o aluno não trabalhará todo o mês de Outubro quando deve iniciar o estágio e Abril quando deve ser finalizado se não for prorrogado.

    \subsection{Funcionalidades Extras}
      Caso todas as atividades acima sejam completadas antes do fim do período de estágio, estas serão abordadas em seguida. O mesmo vale para o caso de uma eventual renovação do aluno.

      \begin{itemize}
        \item Revisão da interface gráfica;
        \item Evolução da listagem priorizada de Configurações;
        \item Permitir cópias (\textit{forks}) de Configurações.
      \end{itemize}

\end{document}