\documentclass[12pt]{article}

\usepackage{url}
\usepackage[brazil]{babel}
\usepackage[utf8]{inputenc}
\usepackage{float}
\usepackage{setspace}
\usepackage{tabu}

\begin{document}
  \title{Plano de Trabalho - Projeto Mezuro\\
         Daniel de Quadros Miranda\\
         NUSP 7577406}

  \maketitle

  \section{Qualificações}
  Por dois anos administrador de sistemas na Rede Linux do IME - USP (\url{http://linux.ime.usp.br}), já possui experiência no projeto Mezuro dentro da disciplina MAC0342 (Laboratório de Programação Extrema) em 2014, sendo assim um candidato muito bom para contribuir ao projeto uma vez que não terá que passar pela curva de aprendizado natural.

  \section{Atividades}
    Vale ressaltar que o projeto adota a metodologias ágeis como programação extrema. Portanto, para estas atividades o aluno terá mais uma pessoa junto. Além da supervisão de alunos mais experientes.

    \subsection{Atividades cotidianas}
      Atividades que serão exercidas todos os dias pelo aluno concorrentemente com o desenvolvimento de funcionalidades.

      \begin{itemize}
        \item Dar manutenção aos serviços do \url{mezuro.org}, resolvendo falhas que apareçam;
        \item Revisar os códigos dos colaboradores externos do projeto.
      \end{itemize}

    \subsection{Novas Funcionalidades Previstas}\label{subsec:func-prev}
      O objetivo principal das funcionalidades pretendidas a serem desenvolvidas pelo aluno em conjunto com o restante do time de desenvolvimento tem o foco principal em melhorar a interação dos usuários finais com o sistema.

      \begin{enumerate}
        \item Integração via login ao novo Portal do Software Público Brasileiro;
        \item Melhorar a automatização da instalação do projeto de forma a atrair mais colaboradores ao projeto;
        \item Entrega contínua em produção.
      \end{enumerate}

      \subsubsection{Cronograma}
        \begin{table}[H]
          \begin{tabu}{| c | c | c | c | c | c | c | c | c |}
            \hline
            \textbf{Mês} & Março & Abril & Maio & Junho & Julho & Agosto & Setembro & Outubro \\ \hline
            \textbf{Atividades} & 1 & 1 & 1 e 2 & 2 & 2 & 2 e 3 & 3 & 3 \\ \hline
          \end{tabu}
          \caption{Cronograma das funcionalidades previstas numeradas de acordo com a lista em \ref{subsec:func-prev}}
        \end{table}

        Vale ressaltar que o estágio é de \textbf{6 meses} e o aluno não trabalhará todo o mês de Março quando deve iniciar o estágio e Outubro quando deve ser finalizado se não for prorrogado.

    \subsection{Funcionalidades Extras}
      Caso todas as atividades acima sejam completadas antes do fim do período de estágio, estas serão abordadas em seguida. O mesmo vale para o caso de uma eventual renovação do aluno.

      \begin{itemize}
        \item Permitir cópias (\textit{forks}) de Configurações;
        \item Permitir o uso de métricas de software Python;
        \item Permitir o uso de métricas de software PHP.
      \end{itemize}


\end{document}