\documentclass[12pt]{article}

\usepackage{url}
\usepackage[brazil]{babel}
\usepackage[utf8]{inputenc}
\usepackage{float}
\usepackage{setspace}
\usepackage{tabu}

\begin{document}
  \title{Plano de Trabalho - Projeto Mezuro\\
         Heitor Reis Ribeiro\\
         NUSP 7577500}

  \maketitle

  \section{Qualificações}
  Já possui experiência no projeto Mezuro de um ano de estágio dentro do mesmo, sendo assim um candidato muito bom para contribuir ao projeto uma vez que não terá que passar pela curva de aprendizado natural que outros membros necessitariam.

  Nesse tempo se tornou um dos principais contribuidores no projeto com atualmente 307 commits contribuídos para o software que hoje se encontra em produção e outros mais no código da próxima versão.

  Além disso, como está no final do curso de bacharelado em ciência da computação, já tem o conhecimento necessário para desenvolver projetos complexos.

  \section{Atividades}
    Vale ressaltar que o projeto adota a metodologias ágeis como programação extrema. Portanto, para estas atividades o aluno terá mais uma pessoa junto. Além da supervisão de alunos mais experientes.

    \subsection{Atividades cotidianas}
      Atividades que serão exercidas todos os dias pelo aluno concorrentemente com o desenvolvimento de funcionalidades.

      \begin{itemize}
        \item Dar manutenção aos serviços do \url{mezuro.org}, resolvendo falhas que apareçam;
        \item Revisar os códigos dos colaboradores externos do projeto.
      \end{itemize}

    \subsection{Novas Funcionalidades Previstas}\label{subsec:func-prev}
      O objetivo principal das funcionalidades pretendidas a serem desenvolvidas pelo aluno em conjunto com o restante do time de desenvolvimento tem o foco principal em melhorar a interação dos usuários finais com o sistema.

      \begin{enumerate}
        \item Desacoplar repositórios de projetos permitindo que o primeiro exista no sistema sem o segundo;
        \item Expandir a gama de métricas Ruby coletadas e adicionar a coleta de métricas para PHP;
        \item Trabalhar na API Python para que os serviços do Kalibro (parte do projeto Mezuro) sejam acessíveis por outras aplicações nessa linguagem, em especial o Colab (parte do novo portal do software público brasileiro).
      \end{enumerate}

      \subsubsection{Cronograma}
        \begin{table}[H]
          \begin{tabu}{| c | c | c | c | c | c | c | c | c |}
            \hline
            \textbf{Mês} & Julho & Agosto & Setembro & Outubro & Novembro & Dezembro\\ \hline
            \textbf{Atividades} & 1 & 1 e 2 & 2 & 2 e 3 & 3 & 3 \\ \hline
          \end{tabu}
          \caption{Cronograma das funcionalidades previstas numeradas de acordo com a lista em \ref{subsec:func-prev}}
        \end{table}

    \subsection{Funcionalidades Extras}
      Caso todas as atividades acima sejam completadas antes do fim do período de estágio, estas serão abordadas em seguida. O mesmo vale para o caso de uma eventual renovação do aluno.

      \begin{itemize}
        \item Permitir cópias (\textit{forks}) de Configurações;
        \item Listagem priorizada de Configurações;
        \item Criação de um subtipo de configuração para métricas de \textit{hotspot}.
      \end{itemize}


\end{document}