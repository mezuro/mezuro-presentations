\begin{abstract}

Despite the ``show me the code'' culture, source
code metrics are often not perceived as an indicator of free software quality.
%
To promote the use of source code metrics, we have defined an approach and
developed a platform to use, interpret, and understand software metrics.
%
Mezuro project aims to improve the readability of source code metrics. It shows
metric results in a friendly way, helping software engineers to spot design
flaws to refactor, project managers to control source code quality, and
software adopters and researchers to compare specific source code
characteristics across free software projects.
%
Mezuro is under development since 2009 and available on the Web from 2013. We
investigated its scalability and found that the most important feature from its
web service was not scalable. We have rewritten the service to solve its issues
and improve the service architecture to bring Mezuro to the new era of cloud
computing with scalability, distributed processing, and fault tolerance through
smaller services.

\end{abstract}

\begin{keywords}
free software, source code metrics, software evolution, micro-service
architecture, cloud computing.
\end{keywords}

