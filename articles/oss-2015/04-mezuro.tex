\newpage
\section{The Mezuro Project}
\label{sec:mezuro}

%TODO: The Mezuro Project (5 páginas)
%% early design (funcionalidades e a forma como foi implementada inicialmente)
%%%% limitations
%% Services-based implementation
%%%% limitations (Scalability evaluation)
%% Cloud-based (explicando que usou micro-services)
%%%% experimental results (mostrar o máximo possível que a coisa funciona e é eficiente, ou seja, atinge os objetivos do artigo).

The Mezuro\footnote{\url{http://mezuro.org}} project aims to be an interface which allows, in a flexible way, the extraction, analysis and interpretation of static source code metrics. Licensed under the Affero General Public License version 3 (AGPLv3), it makes the user responsible for defining the metrics she wants to employ on the analysis, keeping track and providing graphical visualization of the evolution of the selected set of metrics. Its main academical goals are: to get close to a consensus on which set of metrics should be employed on the analysis of different kinds of source code written in different programming languages; to search which interpretation should be given to each value obtained for the set of selected metrics.

TODO: evolucao da arquitetura até chegar em um servico escalavel

TODO: comparar dados de escalabilidade entra as versoes do Kalibro

TODO: exemplo de uso: funcionalidade, demonstrando os resultados 2 projetos de
certo impacto rodando 2 semanas no Mezuro, por exemplo, firefox vs.  chromium
ou gimp vs. inkscape (preferencialmente, não java)
