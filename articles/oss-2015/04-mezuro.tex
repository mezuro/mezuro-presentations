\newpage
\section{The Mezuro Project}
\label{sec:mezuro}

%TODO: The Mezuro Project (5 páginas)
%% early design (funcionalidades e a forma como foi implementada inicialmente)
%%%% limitations
%% Services-based implementation
%%%% limitations (Scalability evaluation)
%% Cloud-based (explicando que usou micro-services)
%%%% experimental results (mostrar o máximo possível que a coisa funciona e é eficiente, ou seja, atinge os objetivos do artigo).

The project Mezuro\footnote{\url{http://mezuro.org}} aims to be a interface which allows in a flexible way the extraction, analysis and interpretation of static source code metrics. Licensed under the Affero General Public License version 3 (AGPLv3), it makes the user responsible for the definition the metrics which it wants to employ on the analysis, storing and providing graphical visualization of the evolution of the selected metric set. Its main academical goals are through the data extracted from user usage: to get close to a consensus on which metrics should be employed on the analysis of each kind of source code; search which interpretation should be given each value obtained for the set of selected metrics.

TODO: evolucao da arquitetura até chegar em um servico escalavel

TODO: comparar dados de escalabilidade entra as versoes do Kalibro

TODO: exemplo de uso: funcionalidade, demonstrando os resultados 2 projetos de
certo impacto rodando 2 semanas no Mezuro, por exemplo, firefox vs.  chromium
ou gimp vs. inkscape (preferencialmente, não java)
