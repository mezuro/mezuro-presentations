\newpage
\section{The Mezuro Project}
\label{sec:mezuro}
%TODO: The Mezuro Project (5 páginas)

The Mezuro\footnote{\url{http://mezuro.org}} project aims to be an interface which allows, in a flexible way, the extraction, analysis and interpretation of static source code metrics. Licensed under the Affero General Public License version 3 (AGPLv3), it makes the user responsible for defining the metrics she wants to employ on the analysis, keeping track and providing graphical visualization of the evolution of the selected set of metrics. From here the set of selected metrics together with the interpretations for each value  will be referenced just as ``configuration''. Its main academical goals are: to get close to a consensus on which configurations should be employed on the analysis of different kinds of source code written in different programming languages and to search which interpretation should be given to each value obtained for the configuration.

\subsection{Early design}
\label{subsec:early-design}
%% early design (funcionalidades e a forma como foi implementada inicialmente)
%%%% limitations

Initially, we developed a desktop application called Kalibro \cite{de2013kalibro}, written in Java which had many of the features we wanted for source code analysis. At that time, Kalibro already supported the selection and composition of metrics to be employed on the analysis and allowed users to define their own interpretation for the results of each metric calculation. With a created configuration, Kalibro only needed an URL for the source code to start the analysis. The URL could be the path for the source code, compressed in a ZIP or TARBALL file, on the user's computer or the link for the repository where the source code was stored. The source code managers supported included GIT, Subversion, CVS, Mercurial and Bazaar. Finally, the source code could be written in Java, C or C++.

Despite Kalibro being capable to successfully analyze source codes, we could not get close to the consensus we aimed for. The main obstacle was that, with a desktop application, it was not possible to incite the discussion on the configurations employed among developers nor the comparison of results between projects. That was the main reason that motivated the move from a desktop application to a service-based implementation.

\subsection{Service-based implementation}
\label{subsec:service-based-implementation}
%% Services-based implementation
%%%% limitations (Scalability evaluation)

The service-based implementation of the Mezuro project was designed to have a back end, that was a monolithic web service that performed all the database and analysis operations, and a front end, that was a plugin of the Noosfero\footnote{\url{http://noosfero.org/Site}} social network. The first one was described in WSDL and communicated with the front end using SOAP messages. It was still written in Java for it was an extension of Kalibro. As for the back end, we wanted that our users were able to discuss and compare their configuration and results, so a social network seemed a good way to achieve that. We did not want to implement a social network from scratch, though. Noosfero was a good choice for its robust plugin support and for being open source. Our plugin was written in Ruby, using the Ruby on Rails framework.

The main advantage of this implementation was to bring the code analysis tools to the web in a way that multiple user could produce their own configurations, view and share historical results. In the other hand the monolithic web service was too big and the technologies used outdated making it difficult for maintenance, development and scalability. So mainly with the objective of splitting the monolith and updating the technologies, all the code were rewritten using the latest version of the Ruby on Rails framework.

\subsection{Cloud-based implementation}
\label{subsec:cloud-based-implementation}
%% Cloud-based (explicando que usou micro-services)
%%%% experimental results (mostrar o máximo possível que a coisa funciona e é eficiente, ou seja, atinge os objetivos do artigo).

The rewritten code was split into three different services: Prezento; Kalibro Processor; and Kalibro Configurations. The user interface, code analyzer and configuration manager respectively. Those services communicates through HTTP requests encapsulated onto a fourth piece of software called KalibroClient.

All this work is based on the micro-service architecture \cite{namiot2014micro}. With this new distributed architecture the platform has more flexibility to scale into the cloud. And, as well, demands less knowledge from developers about the existing code, since they should not know all the services but just the one which relates to the feature getting implemented and its API.
