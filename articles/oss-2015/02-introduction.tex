\section{Introduction}
\label{introduction}

% TODO:
% introduction (1 página)
%% métricas de código fonte são importantes e úteis para X (apesar de que é uma coisa limitado)
%% já existem ferramentas que fazer K e L e são boas
%% Não existem atualmente ferramentas apropriadas para Y (as ferramentas atuais para métricas apresentam limitações Z e W)
%% Neste artigo, apresentamos o processo de evolução da ferramenta Mezuro para o ambiente de computação em nuvem que resolve o problema de uma forma inovadora.

Whatever the methodology, from a practical point of view, software development
should be guided by an aspects to control the software quality in the long
term: source code quality~\cite{martin2008}. Metrics can help software
engineers to observe the source code quality~\cite{SEI88}. Software Engineering
requires the understanding of software, which is the result from the writing of
source code~\cite{martin2008}. Also, we argue that software engineers and
researchers need to analyze source codes to understand software projects.

For free software communities~\footnote{\url{gnu.org/philosophy/free-sw.html}}
\footnote{We consider the terms ``free software'' and ``open source software''
equivalent.}, source code is the main artefact of software development
activities since features should be constantly released to users. In fact,
free software source code is written gradually and different developers make
updates as well as improvements on an ongoing basis~\cite{martin2008}. Thus,
new features are inserted and bugs are resolved during software development and
maintenance iterations.

Software engineers make decisions when are programming at the method and class
level, influencing the source code quality~\cite{beck2007}. To make the best
decisions, we argue they should track attributes of their source codes from an
automated and objective way to interpret metric values. Even with the fact that
source code metrics have been proposed since the 1970s~\cite{SEI88}, there is
not a set of standard measures established for them.  Moreover, we have
observed there is not a systematic approach to use, interpret, and understand
source code metrics. We have identified limitations from several FLOSS
tools to monitor this kind of metrics, such as for collecting automatically
source code metrics values independent of the programming language, for
interpreting measurement results associating them with source code quality, and
an avaliable and dependable service on the cloud.

In this paper, we present Mezuro, a free software platform
designed to incorporate any source code metric tool, extending it to provide
easy to understand evaluation of the analyzed software quality. 
TODO: evolving and cloud context...

The remainder of this paper is organized as follows: Section
\ref{sec:related_works} describes related work.  Section \ref{sec:mezuro}
presents the Mezuro project and how its implementation evolved from a desktop
application to a distributed system that runs on the cloud.  At last, on
section \ref{sec:conclusions} we conclude the paper and discuss future work.

