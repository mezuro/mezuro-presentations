\documentclass[12pt]{article}

\usepackage{sbc-template}
\usepackage{graphicx,url}
\usepackage[brazil]{babel}   
\usepackage[latin1]{inputenc}  
     
\sloppy

\title{Como a Ado��o de Projetos de Software Livre Como Base para Novos Projetos Afeta Seu Desenvolvimento}

\author{Rafael R. Manzo\inst{1}, Diego de A. M. Camarinha\inst{1},\\
        Guilherme H. R. V. de Lima\inst{1}, Fellipe S. Sampaio\inst{1},\\
        Renan Fichberg\inst{1}, Paulo Meirelles\inst{2}, Fabio Kon\inst{1}}


\address{Instituto de Matem�tica e Estat�stica -- Universidade de S�o Paulo (USP)\\
  Rua do Mat�o, 1010 -- 05508-090 -- Cidade Universit�ria -- S�o Paulo -- SP -- Brasil
\nextinstitute
  Faculdade de Engenharia -- UnB Gama (FGA)\\
  Gama -- DF -- Brasil 
  \email{\{manzo,kon\}@ime.usp.br,\{diego.camarinha,guilherme.henrique.lima\}@usp.br}
  \email{\{renan.fichberg,fellipe.sampaio\}@usp.br,paulo@softwarelivre.org}
}

\begin{document} 

\maketitle

\begin{abstract}
  This resumed article approaches one of the most important decisions when starting a new free software project: to adopt and base the new project in a old and madure one; or to create a whole new and independent one. Will be discussed the vantages and disavantages for each option. For, finally, present a real experience from a project that started by choosing the second option, during it's early development choose to adopt the first one, to  recently again choose the second one.
\end{abstract}
     
\begin{resumo} 
  Este artigo resumido trata sobre uma das decis�es mais importantes no momento de se iniciar um novo projeto de software livre: adotar como base outro projeto j� maduro; ou escrever um totalmente novo e independente. Ser�o discutidas brevemente as vantagens e desvantagens de cada op��o. Para, por fim, apresentar uma experi�ncia real de um projeto que teve in�cio decidindo pela segunda op��o, durante o in�cio de seu desenvolvimento resolveu adotar a segunda op��o, para recentemente novamente adotar a primeira op��o.
\end{resumo}


\section{Introdu��o} \label{sec:introducao}

\section{Decis�o ao se iniciar um novo projeto} \label{sec:decisao}
  \subsection{Projeto existente como base} \label{subsec:projeto-existente}

  \subsection{Projeto totalmente novo} \label{subsec:projeto-novo}

\section{Projeto Mezuro} \label{sec:projeto-mezuro}
  \subsection{Hist�ria} \label{subsec:historia}
  \subsection{Estado atual} \label{subsec:estado-atual}
  \subsection{Li��es aprendidas} \label{subsec:licoes-aprendidas}

\section{Conclus�o} \label{sec:conclusao}

\bibliographystyle{sbc}
\bibliography{impacto-software-livre}

\end{document}
