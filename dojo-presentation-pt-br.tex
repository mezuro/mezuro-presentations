\documentclass{beamer}
\usepackage[T1]{fontenc}
\usepackage[utf8]{inputenc}
\usepackage{lmodern}
\usepackage[brazil]{babel}

\usetheme{JuanLesPins}

\title{
       \textbf{Dojo de programação} \\
       CCSL - IME - USP
      }

\begin{document}

\maketitle

\section{Introdução}
\begin{frame}
  \frametitle{Mas o que é dojo?}
  \framesubtitle{}

  Você já viu algum programador \textbf{treinando}? \\~\\

  A alma de um dojo de programação é essa:
  \begin{itemize}
    \item Escrever o \textbf{melhor} código que você puder;
    \item \textbf{Sem compromisso} de finalizar;
    \item Trocar \textbf{experiências} com outros programadores.
  \end{itemize}
\end{frame}

\begin{frame}
  \frametitle{Funciona assim...}

  \begin{itemize}
    \item Um monte de \textbf{programadores} (ou curiosos) se juntam;
    \item Escolhem um \textbf{desafio} (ou tema) sobre o qual querem aprender;
    \item \textbf{Pares} programam enquanto os demais prestam atenção na solução sendo implementada;
      \begin{itemize}
        \item \textbf{Rotação}: O par é trocado a cada poucos minutos;
        \item \textbf{Testes}: Queremos código de qualidade, então fazemos TDD (Test Driven Development);
        \item \textbf{Não} há obrigação de \textbf{finalizar} o desafio.
      \end{itemize}
    \item Depois que encerramos, conversamos sobre como foi tudo e como podemos \textbf{melhorar}.
  \end{itemize}
\end{frame}

\section{Temas}
\begin{frame}
  \frametitle{Escolha do desafio}

  \begin{itemize}
    \item Internacionalização com RoR (Ruby on Rails);
    \item Criação de modelos sem ActiveRecord no RoR;
    \item TODO: olhar código do kalibro entities e ver uns temas...
  \end{itemize}
\end{frame}

\begin{frame}
  \frametitle{Escolha do desafio}
  \framesubtitle{Internacionalização}

  Uma mesma aplicação pode ser acessada por pessoas com \textbf{diferentes preferências de idioma} além do inglês. \\~\\

  Então, pensando nisso, nossos amigos desenvolvedores do Rails incorpararam a gem \textbf{i18n} ao framework, facilitando nossa vida. \\~\\

  Que tal internacionalizarmos algumas partes do novo Mezuro?
\end{frame}

\begin{frame}
  \frametitle{Escolha do desafio}
  \framesubtitle{nonActiveRecord models}

  Em Rails todos os modelos clássicos herdam de uma classe chamada ActiveModel, parte da gem ActiveRecord que precisa de um banco de dados. \\~\\

  E se você \textbf{não} quiser um \textbf{banco de dados}, mas quer todas as outras coisas legais que um modelo do Rails tem? \\~\\

  No Mezuro demos um jeito nisso! Quer ver o que fizemos e dar sugestões?
\end{frame}

\begin{frame}
  \frametitle{Escolha do desafio}
  \framesubtitle{Desenvolvimento de gems para Ruby}

  TODO
\end{frame}

\begin{frame}
  \LARGE{\textbf{Vamos votar!}}
\end{frame}

\section{Atividade}
\begin{frame}
  \LARGE{\textbf{Hora de programar então}}
\end{frame}

\section{Retrospectiva}
\begin{frame}
  \framesubtitle{Foi bom para você?}

  \begin{itemize}
    \item É livre para você dizer o que quiser;
    \item Sugestões são muito bem vindas;
    \item E críticas são ainda mais!
  \end{itemize}
\end{frame}

\begin{frame}
  \LARGE{\textbf{ACABOU :(}} \\~\\

  Mantenha contato conosco:
  \textbf{manzo@ime.usp.br}
\end{frame}
\end{document}
