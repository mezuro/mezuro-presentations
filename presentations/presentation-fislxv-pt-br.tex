\documentclass{beamer}
\usepackage[T1]{fontenc}
\usepackage[utf8]{inputenc}
\usepackage{lmodern}
\usepackage[brazil]{babel}

\usetheme{JuanLesPins}

\title{
       \textbf{Como podemos e devemos monitorar nosso código-fonte} \\
       CCSL - IME - USP\\
       FGA - UnB
      }
\author{
        Arthur Del Esposte \\
        Diego Araújo \\
        Paulo Meirelles \\
        Rafael Manzo
       }

\begin{document}

\maketitle

\section{Introdução}
  \subsection{Métricas}
  \begin{frame}
    \frametitle{O que são métricas?}
    \framesubtitle{}

    \begin{itemize}
      \item Medidas extraídas do software que fornecem informações sobre sua
        \begin{itemize}
          \item complexidade
          \item capacidade de ser compreendido
          \item testabilidade
          \item manutenabilidade
          \item evolução
        \end{itemize}
      \item Dois tipos
        \begin{itemize}
          \item \textbf{Estáticas}: apenas o fazem análise léxica e sintática código-fonte
          \item Dinâmicas\footnote{Por exemplo, cobertura de testes}: necessitam que o código esteja compilado ou seja executado de alguma forma
        \end{itemize}
    \end{itemize}
  \end{frame}

  \begin{frame}
    \frametitle{Exemplos de métricas}
    \framesubtitle{Primitivas}
  \end{frame}

  \begin{frame}
    \frametitle{Exemplos de métricas}
    \framesubtitle{Compostas}
  \end{frame}

\section{Motivação}
\begin{frame}
  \frametitle{}
  \framesubtitle{}

  Motivação
\end{frame}

\begin{frame}
  \frametitle{Crise do software}
  \framesubtitle{O problema}
\end{frame}

\begin{frame}
  \frametitle{Crise do software}
  \framesubtitle{Usando métricas}
\end{frame}

\begin{frame}
  \frametitle{Crise do software}
  \framesubtitle{Quais métricas? Como interpretar?}
\end{frame}

\section{Soluções similares}
\begin{frame}
  \frametitle{}
  \framesubtitle{}

  Soluções similares
\end{frame}

\begin{frame}
  \frametitle{Sonar}
  \framesubtitle{}
\end{frame}

\begin{frame}
  \frametitle{Code Climate}
  \framesubtitle{}
\end{frame}

\section{Mezuro}
\begin{frame}
  \frametitle{}
  \framesubtitle{}

  Mezuro
\end{frame}

\begin{frame}
  \frametitle{Ideais}
  \framesubtitle{}
\end{frame}

\begin{frame}
  \frametitle{Breve história}
  \framesubtitle{}
\end{frame}

\begin{frame}
  \frametitle{Arquitetura}
  \framesubtitle{}
\end{frame}

\begin{frame}
  \frametitle{Principais funcionalidades}
  \framesubtitle{}
\end{frame}

  \subsection{Demonstração}
  \begin{frame}
    \frametitle{Criação e processamento de repositório}
    \framesubtitle{Projeto}
  \end{frame}

  \begin{frame}
    \frametitle{Criação e processamento de repositório}
    \framesubtitle{Repositório}
  \end{frame}

  \begin{frame}
    \frametitle{Criação e processamento de repositório}
    \framesubtitle{Exibição de resultados}
  \end{frame}

  \begin{frame}
    \frametitle{Criação de configuração}
    \framesubtitle{Configuração}
  \end{frame}

  \begin{frame}
    \frametitle{Criação e processamento de repositório}
    \framesubtitle{Escolha de métrica e interpretação}
  \end{frame}

\section{Conclusão}
\begin{frame}
  \frametitle{Conclusão}
  \framesubtitle{}

  Conclusão
\end{frame}

\begin{frame}
  \frametitle{Conclusão}
  \framesubtitle{Revisão}
\end{frame}

\begin{frame}
  \frametitle{Conclusão}
  \framesubtitle{Próximos passos}
\end{frame}

\begin{frame}
  \frametitle{Conclusão}
  \framesubtitle{Nos acompanhe}

  Obrigado!
\end{frame}
\end{document}